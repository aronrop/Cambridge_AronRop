%!TEX root = ../thesis.tex
%*******************************************************************************
%****************************** Third Chapter **********************************
%*******************************************************************************
\chapter{My third chapter}

% **************************** Define Graphics Path **************************
\ifpdf
    \graphicspath{{Chapter3/Figs/Raster/}{Chapter3/Figs/PDF/}{Chapter3/Figs/}}
\else
    \graphicspath{{Chapter3/Figs/Vector/}{Chapter3/Figs/}}
\fi

\section{First section of the third chapter}
And now I begin my third chapter here \dots

And now to cite some more people~\cite{Rea85,Ancey1996}

\subsection{First subsection in the first section}
\dots and some more 

\subsection{Second subsection in the first section}
\dots and some more \dots

\subsubsection{First subsub section in the second subsection}
\dots and some more in the first subsub section otherwise it all looks the same
doesn't it? well we can add some text to it \dots

\subsection{Third subsection in the first section}
\dots and some more \dots

\subsubsection{First subsub section in the third subsection}
\dots and some more in the first subsub section otherwise it all looks the same
doesn't it? well we can add some text to it and some more and some more and
some more and some more and some more and some more and some more \dots

\subsubsection{Second subsub section in the third subsection}
\dots and some more in the first subsub section otherwise it all looks the same
doesn't it? well we can add some text to it \dots

\section{Second section of the third chapter}
and here I write more \dots

\section{The layout of formal tables}
This section has been modified from ``Publication quality tables in \LaTeX*''
 by Simon Fear.

The layout of a table has been established over centuries of experience and 
should only be altered in extraordinary circumstances. 

When formatting a table, remember two simple guidelines at all times:

\begin{enumerate}
  \item Never, ever use vertical rules (lines).
  \item Never use double rules.
\end{enumerate}

These guidelines may seem extreme but I have
never found a good argument in favour of breaking them. For
example, if you feel that the information in the left half of
a table is so different from that on the right that it needs
to be separated by a vertical line, then you should use two
tables instead. Not everyone follows the second guideline:

There are three further guidelines worth mentioning here as they
are generally not known outside the circle of professional
typesetters and subeditors:

\begin{enumerate}\setcounter{enumi}{2}
  \item Put the units in the column heading (not in the body of
          the table).
  \item Always precede a decimal point by a digit; thus 0.1
      {\em not} just .1.
  \item Do not use `ditto' signs or any other such convention to
      repeat a previous value. In many circumstances a blank
      will serve just as well. If it won't, then repeat the value.
\end{enumerate}

A frequently seen mistake is to use `\textbackslash begin\{center\}' \dots `\textbackslash end\{center\}' inside a figure or table environment. This center environment can cause additional vertical space. If you want to avoid that just use `\textbackslash centering'


\begin{table}
\caption{A badly formatted table}
\centering
\label{table:bad_table}
\begin{tabular}{|l|c|c|c|c|}
\hline 
& \multicolumn{2}{c}{Species I} & \multicolumn{2}{c|}{Species II} \\ 
\hline
Dental measurement  & mean & SD  & mean & SD  \\ \hline 
\hline
I1MD & 6.23 & 0.91 & 5.2  & 0.7  \\
\hline 
I1LL & 7.48 & 0.56 & 8.7  & 0.71 \\
\hline 
I2MD & 3.99 & 0.63 & 4.22 & 0.54 \\
\hline 
I2LL & 6.81 & 0.02 & 6.66 & 0.01 \\
\hline 
CMD & 13.47 & 0.09 & 10.55 & 0.05 \\
\hline 
CBL & 11.88 & 0.05 & 13.11 & 0.04\\ 
\hline 
\end{tabular}
\end{table}

\begin{table}
\caption{A nice looking table}
\centering
\label{table:nice_table}
\begin{tabular}{l c c c c}
\hline 
\multirow{2}{*}{Dental measurement} & \multicolumn{2}{c}{Species I} & \multicolumn{2}{c}{Species II} \\ 
\cline{2-5}
  & mean & SD  & mean & SD  \\ 
\hline
I1MD & 6.23 & 0.91 & 5.2  & 0.7  \\

I1LL & 7.48 & 0.56 & 8.7  & 0.71 \\

I2MD & 3.99 & 0.63 & 4.22 & 0.54 \\

I2LL & 6.81 & 0.02 & 6.66 & 0.01 \\

CMD & 13.47 & 0.09 & 10.55 & 0.05 \\

CBL & 11.88 & 0.05 & 13.11 & 0.04\\ 
\hline 
\end{tabular}
\end{table}


\begin{table}
\caption{Even better looking table using booktabs}
\centering
\label{table:good_table}
\begin{tabular}{l c c c c}
\toprule
\multirow{2}{*}{Dental measurement} & \multicolumn{2}{c}{Species I} & \multicolumn{2}{c}{Species II} \\ 
\cmidrule{2-5}
  & mean & SD  & mean & SD  \\ 
\midrule
I1MD & 6.23 & 0.91 & 5.2  & 0.7  \\

I1LL & 7.48 & 0.56 & 8.7  & 0.71 \\

I2MD & 3.99 & 0.63 & 4.22 & 0.54 \\

I2LL & 6.81 & 0.02 & 6.66 & 0.01 \\

CMD & 13.47 & 0.09 & 10.55 & 0.05 \\

CBL & 11.88 & 0.05 & 13.11 & 0.04\\ 
\bottomrule
\end{tabular}
\end{table}
\subsection{ctable}
You can use ctable package to create highly customizable tables. Check documentation for further explanation and see~\cref{nowidth} 

\ctable[
cap = The Skewing Angles,
caption = The Skewing Angles ($\beta$) for
$\fam0 Mu(H)+X_2$ and $\fam0 Mu(H)+HX$~\tmark,
label = nowidth,
pos = h
]{rlcc}{
	\tnote{for the abstraction reaction,
		$\fam0 Mu+HX \rightarrow MuH+X$.}
	\tnote[b]{1 degree${} = \pi/180$ radians.}
	\tnote[c]{this is a particularly long note, showing that
		footnotes are set in raggedright mode as we don't like
		hyphenation in table footnotes.}
}{ \FL
	& & $\fam0 H(Mu)+F_2$ & $\fam0 H(Mu)+Cl_2$ \ML
	&$\beta$(H) & $80.9^\circ$\tmark[b] & $83.2^\circ$ \NN
	&$\beta$(Mu) & $86.7^\circ$ & $87.7^\circ$ \LL
}



\subsection{lenghty table that cover whole width}
In case of double column layout, tables which do not fit in single column width should be set to full text width. For this, you need to use \verb+\begin{table*}+ \verb+...+ \verb+\end{table*}+ instead of \verb+\begin{table}+ \verb+...+ \verb+\end{table}+ environment. confirm if footnotes are getting printed and find a fix if not or convert to ctable
\begin{table}[h]
	\caption{Example of a lengthy table which is set to full textwidth}\label{tab2}
	\begin{tabular*}{\textwidth}{@{\extracolsep\fill}lcccccc}
		\toprule%
		& \multicolumn{3}{@{}c@{}}{Element 1\footnotemark[1]} & \multicolumn{3}{@{}c@{}}{Element 2\footnotemark[2]} \\\cmidrule{2-4}\cmidrule{5-7}%
		Project & Energy & $\sigma_{calc}$ & $\sigma_{expt}$ & Energy & $\sigma_{calc}$ & $\sigma_{expt}$ \\
		\midrule
		Element 3  & 990 A & 1168 & $1547\pm12$ & 780 A & 1166 & $1239\pm100$\\
		Element 4  & 500 A & 961  & $922\pm10$  & 900 A & 1268 & $1092\pm40$\\
		\bottomrule
	\end{tabular*}
	\footnotetext{Note: This is an example of table footnote. This is an example of table footnote this is an example of table footnote this is an example of~table footnote this is an example of table footnote.}
	\footnotetext[1]{Example for a first table footnote.}
	\footnotetext[2]{Example for a second table footnote.}
\end{table}










\subsection{sideways table}
You can use sideways environment to create sideways table to carry more data in a single page. 
\begin{sidewaystable}
	\renewcommand{\thefootnote}{\alph{footnote}}%to force alpahbet tmark
	\caption{Tables which are too long to fit, should be written using the ``sidewaystable'' environment as shown here}\label{tab3}
	\begin{tabular*}{\textheight}{@{\extracolsep\fill}lcccccc}
		\toprule%
		& \multicolumn{3}{@{}c@{}}{Element 1\footnotemark[1]}& \multicolumn{3}{@{}c@{}}{Element\footnotemark[2]} \\\cmidrule{2-4}\cmidrule{5-7}%
		Projectile & Energy	& $\sigma_{calc}$ & $\sigma_{expt}$ & Energy & $\sigma_{calc}$ & $\sigma_{expt}$ \\
		\midrule
		Element 3 & 990 A & 1168 & $1547\pm12$ & 780 A & 1166 & $1239\pm100$ \\
		Element 4 & 500 A & 961  & $922\pm10$  & 900 A & 1268 & $1092\pm40$ \\
		Element 5 & 990 A & 1168 & $1547\pm12$ & 780 A & 1166 & $1239\pm100$ \\
		Element 6 & 500 A & 961  & $922\pm10$  & 900 A & 1268 & $1092\pm40$ \\
		\bottomrule
	\end{tabular*}
	\footnotetext{Note: This is an example of table footnote this is an example of table footnote this is an example of table footnote this is an example of~table footnote this is an example of table footnote.}
	\footnotetext[1]{This is an example of table footnote.}
	\footnotetext[2]{A very important table footnote}
\end{sidewaystable}

\section{Equations}\label{sec4}

Equations in \LaTeX\ can either be inline or on-a-line by itself (``display equations''). For
inline equations use the \verb+$...$+ commands. E.g.: The equation
$H\psi = E \psi$ is written via the command \verb+$H \psi = E \psi$+.

For display equations (with auto generated equation numbers)
one can use the equation or align environments:
\begin{equation}
	\|\tilde{X}(k)\|^2 \leq\frac{\sum\limits_{i=1}^{p}\left\|\tilde{Y}_i(k)\right\|^2+\sum\limits_{j=1}^{q}\left\|\tilde{Z}_j(k)\right\|^2 }{p+q}.\label{eq1}
\end{equation}
where,
\begin{align}
	D_\mu &=  \partial_\mu - ig \frac{\lambda^a}{2} A^a_\mu \nonumber \\
	F^a_{\mu\nu} &= \partial_\mu A^a_\nu - \partial_\nu A^a_\mu + g f^{abc} A^b_\mu A^a_\nu \label{eq2}
\end{align}
Notice the use of \verb+\nonumber+ in the align environment at the end
of each line, except the last, so as not to produce equation numbers on
lines where no equation numbers are required. The \verb+\label{}+ command
should only be used at the last line of an align environment where
\verb+\nonumber+ is not used.
\begin{equation}
	Y_\infty = \left( \frac{m}{\textrm{GeV}} \right)^{-3}
	\left[ 1 + \frac{3 \ln(m/\textrm{GeV})}{15}
	+ \frac{\ln(c_2/5)}{15} \right]
\end{equation}

